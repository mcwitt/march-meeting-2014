Recently it has been proposed~[1] that the Graph Isomorphism (GI) problem could
be solved using a quantum annealer. This is done by encoding the graphs into
Ising Hamiltonians, identifying the vertices with spins and the edges with
antiferromagnetic interactions. The idea is that measurements of simple
observables during and at the end of the annealing process should distinguish
non-isomorphic graphs. The first experimental study of the GI problem using
D-Wave's quantum computer has been carried out by Vinci et al.~[2], utilizing
measurements taken at the end of the annealing process. Here, we will present
preliminary evidence that measurements taken part way through the annealing
process, now obtainable using state-of-the-art devices, may offer better
distinguishing capabilities.\\[4pt]
[1] I. Hen and A. P. Young, Physical Review A \textbf{86} (2012).\\[0pt]
[2] W. Vinci et al., arXiv:1307.1114.
